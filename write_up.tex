\documentclass[12pt, article]{scrartcl}
\usepackage[english]{babel}
\usepackage{sectsty}
\allsectionsfont{\centering \normalfont\scshape}
\usepackage{fancyhdr}
\pagestyle{fancyplain}
\fancyhead{}
\fancyfoot[L]{}
\fancyfoot[C]{}
\fancyfoot[R]{\thepage}
\renewcommand{\headrulewidth}{0pt}
\renewcommand{\footrulewidth}{0pt}
\setlength{\headheight}{13.6pt}
\newcommand{\horrule}[1]{\rule{\linewidth}{#1}}

\title{	
\normalfont \normalsize 
\textsc{University of Idaho: CS 470 - Artificial Intelligence} \\ [25pt]
\horrule{0.5pt} \\[0.4cm]
\huge Project 3b: Moonlander using Fuzzy Logic\\
\horrule{2pt} \\[0.5cm]
}
\author{Andrew Schwartzmeyer}
\date{\normalsize\today}

\begin{document}
\maketitle 
\begin{abstract}
For this project I wrote a Python program to simulate a ``moonlander'' with varying environmental conditions and two dimensions of movement. An artificial intelligence in the form of a fuzzy logic control system was implemented to control the moonlander. As per the project's requirements, the moonlander has two directional rockets, a burner and a thruster, that adjust its position vertically and horizontally respectively. The burner is unidirectional, that is, it can only slow a descent, not speed it up; however, the thruster can adjust the craft in either direction horizontally. The craft's ``environment'' is represented by a downward acceleration, vertical and horizontal position, an amount of fuel, and a strictly horizontal ``wind'' (which is a constant force; no air resistance is taken into account at any point). The wind, acceleration, and initial vertical velocity are randomized with each new craft. The $x$ and $y$ positions and velocities are sent as inputs to the fuzzy control system. After mapping these into fuzzy sets and making a recommendation, the results are then defuzzified and sent back to the craft as burn and thrust amounts. To demonstrate the fuzzy control system, $100,000$ moonlanders and environments were simulated, with a success ful landing rate (within the constraints of the assignment) of $95.6$ percent.
\end{abstract}
\pagebreak
\section{Algorithm}
Project 3b focussed on using fuzzy logic to create an automatic controller. Instead of dealing with exact values, ``fuzzy numbers'' are used, where each fuzzy number is defined by a fuzzy set. In particular, each input to the moonlander was mapped among three fuzzy sets, for a total of twelve. These fuzzy number objects (implemented using the FuzzPy library) have a method \emph{mu} which returns the \mu, or membership level, of any value in the universal set domain of the fuzzy number. The fuzzy numbers used in this project were all trapezoidal sets, with a pair of tuples defining the kernels and supports of the trapezoid, although a few have zero-length kernels, thus becoming triangular.

To define the fuzzy logic rules, I used fuzzy associative matrices. Two three-by-three matrices were created, one for each output/pair of inputs, for a total of eighteen fuzzy AND based rules. Each of the two sets of rules were defuzzified using a weighted average approach, where each rule's output (that is, the minimum \mu value of an intersection on the matrix), was multiplied by a hand-calibrated weight, then added to the other rules, then divided by the total \mu values. The final output was then returned to the moonlander.
\section{Results}
The final success rate after hand-calibration of the defuzzification method's weights was calculated at $95.6$ percent over $100,000$ randomly generated simulations. On average, $51.2$ units of fuel were consumed, the final landing velocity averaged to $2.93$ units of distance per tick, and the final landing position was on average $0.064$ units of distance away from the center of the landing site. The allowed maximum landing speed was $4$ units per tick, and the maximum safe distance from the center was $0.2$ units of distance. Compared to the artificial neural network from Project 3a, the fuzzy controller was far more successful, especially considering the neural network never successfully landed. The program did not have any particular strengths or weaknesses that I could discern from the gathered data. Almost all failed landings were due to slightly too high final vertical velocity. This could likely be accounted for with further refinement to the defuzzification weights, or with adjusted fuzzy sets.
\section{Conclusion}
\end{document}
